\documentclass[main.tex]{subfiles}
\begin{document}

\section{Programing Basics} 
\label{sec:programing_basics}

\begin{frame}{Hello World}
	Let's start with a hello world program. See \textit{helloWorld.c}
	\begin{itemize}
		\item What's the main thing?
		\item Let's break something!
		\item Questions?
	\end{itemize}
\end{frame}

\section{Variables}
\label{sec:variables_scope_and_arrays}

\begin{frame}[fragile]{Variables - a Review}
	\begin{itemize}
		\item Containers for memory - why do we use variables in math?
		\item Types: What kind of data am I using?
		\item Syntax:
			\begin{lstlisting}
//a declaration
<type> <name>; 

//an instanciation 
// (which means "instance creation")
<name> = <value>;

//both at the same time
<type> <name>  = <value>;
			\end{lstlisting}
	\end{itemize}
\end{frame}

\begin{frame}[fragile]{Operators - Logical}
	AKA boolean operators.
	\begin{lstlisting}
		char a = 0;
		int b = 1;

		char x = 0 && 1;
		char y = b || a;
		char z = a^b;
		char n = !a;
	\end{lstlisting}
\end{frame}

\begin{frame}[fragile]{Operators - Arithmetic}
	Let's play with some operators!
	\begin{lstlisting}
int a = 1; 
double b = 2;

int c = a + b;
double x = a / b;
long y = a*b;
char z = a % b;
	\end{lstlisting}

	\begin{itemize}
		\item Try changing the types. 
		\item Chain multiple operators in one statement.
		\item Try doing weird things like divide by zero.
	\end{itemize}
\end{frame}

\begin{frame}[fragile]{Operators - Binary}
	\begin{lstlisting}

char c = ~a;
char x = 1<<3;
long y = 7 & 1<<2;
char z = x | 5;
	\end{lstlisting}

	\begin{itemize}
		\item Try changing the types. 
		\item Chain multiple operators in one statement.
	\end{itemize}
\end{frame}

% section variables (end)

\section{Functions} 
\label{sec:functions}

\begin{frame}{Functions}
	\begin{itemize}
		\item Functions are used to modularize code
		\item Variables must be initialized at the top of the function.
	\end{itemize}
\end{frame}

\begin{frame}{Functions}
	\begin{itemize}
		\only<1->{
			\item Function in math: $\text{name}(\text{args}) = \text{stuff}$
			}
		\only<2->{\item Example: $foo(x,y) = x+y$}
		\only<3->{\item Some basic function syntax:}
	\end{itemize}
\end{frame}

\begin{frame}[fragile]{Functions - An Example}
	Function Syntax
	\begin{lstlisting}
<type> name(<args>)
{

}
	\end{lstlisting}
\end{frame}

\begin{frame}[fragile]{Functions - An Example}
		\begin{lstlisting}
double foo(double x, double y)
{
	return x+y;
}
		\end{lstlisting}
\end{frame}

\begin{frame}[fragile]{Functions and Scope}
Scope decides which variable to use when its used within some piece of code.

	\begin{lstlisting}[language=c]

int x = 0;

int f(int x)
{
	//which x is being used to calculate 
	//the result of this function?
	return (x+2);
}

int main(char argc, char** argv)
{
	int x = 0;
}

	\end{lstlisting}
\end{frame}

\begin{frame}[fragile]{Functions and Scope}
	Rule of Thumb!
	\begin{itemize}
		\item Never use global variables
		\item Why? \only<2->{You may never know what kind of data is in it} 
	\end{itemize}
	Remember: All rules can be broken given the right circumstances.
\end{frame}

% section functions (end)

\section{Arrays}
\label{sec:arrays}

\begin{frame}[fragile]{Arrays}
	\begin{itemize}
		\item Arrays - how we represent sets of homogenous data.
		\item Syntax:
			\begin{lstlisting}
<type> <name> [<size>];

<name>[<index>] = <value>;
			\end{lstlisting}
		\item why do indeces begin at 0?
	\end{itemize}
\end{frame}

\begin{frame}[fragile]{Arrays - Behind the Scenes}
	What Arrays Really (But Not Really) look like

	\begin{figure}[H]
		\begin{center}
			\includegraphics[width=0.5\linewidth]{arrayPic.jpg}
		\end{center}
	\end{figure}

\end{frame}

\begin{frame}[fragile]{How To Pass Arrays To Functions}
	We use special things call pointers. Use Google to learn more about them.
	Practice with them is the best way to understand them.

	\begin{lstlisting}
		int func(int* xs, size_t xsSize)
		{
			... 
			xs[0] = ... ;
			...

			return 0;
		}
	\end{lstlisting}
\end{frame}

\begin{frame}[fragile]{Arrays - Rules of Arrays}

	\begin{itemize}
		\item Don't return arrays. Just modify them...or pass in another array to
			store the result. 
		\item \textbf{Don't forget to pass in the array size as a parameter!}
		\item Always check to make sure an array isn't null: 
			\begin{lstlisting}
				...
				if(xs == NULL)
					return 1;
				...
				
			\end{lstlisting}
	\end{itemize}

\end{frame}


\end{document}	
